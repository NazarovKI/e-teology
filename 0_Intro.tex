\tableofcontents
\newpage
\large
\section*{Введение}
Актуальность процесса цифровизации профессионального образования и обучения вызвана глобальными процессами перехода к цифровой экономике и цифровому обществу.
Приоритет построения цифрового образования зафиксирован в федеральных стратегических документах\cite{fsd}. 
Актуальность преподавания ОРКСЭ В системе образования теология, выполняет важную мировоззренческую и воспитательную функции, имеет тесную связь с литературой, историей, философией и другими предметами, формирующими мировоззрение человека, помогает установить связи между естественными и гуманитарными науками.
На фоне того, что методы преподавания теологических дисциплин находятся в процессе становления, общее повышение качества образовательного процесса должно затронуть и качество преподавания теологических дисциплин.
Предполагается, что построение единого информационного пространства в сфере образовании при правильном подходе позволит повысить эффективность и качество процесса обучения за счёт:
\begin{itemize}
\item интенсификации научных изысканий в образовательных учреждениях;
\item сокращения время и улучшения условий для профессионального образования;
\item повышения оперативности и эффективности управления отдельными образовательными учреждениями и сферой образования в целом;
\item интеграции национальных образовательных ИКТ в международную сеть информационных ресурсов и облегчения доступа к ним всех желающих.
\end{itemize}

Процесс перехода от <<традиционных>> форм к современным встречает определённые проблемы, требующие выявления потребностей, постановки задач и получения новых знаний для их реализации.

Объект исследования: \label{obj} существующие электронные образовательные ресурсы (далее – ЭОР) по теологическим дисциплинам.

Предмет исследования: \label{subj} педагогический потенциал электронных образовательных ресурсов в преподавании основ религиозной культуры и светской этики (далее – ОРСКЭ). 

Цель работы:\label{goal}  выявление педагогического потенциала существующих ЭОР для преподавания модуля «Основы православной культуры» (далее – ОПК). 

Задачи:
\begin{enumerate}
\item\label{task1} раскрыть сущность и особенности ЭОР;
\item\label{task2}  рассмотреть требования, предъявляемые к ЭОР и инструменты для проектирования; 
\item\label{task3}  проанализировать потребность в ЭОР в преподавании теологических дисциплин;
\item\label{task4}  определить задачи электронных образовательных ресурсов для преподавания ОРКСЭ (модуль «Основы православной культуры») и предложить проект образовательного ресурса.

\end{enumerate}


Методы исследования – общенаучные: описание, классификация, анализ, синтез, оценка, логическое проектирование.

Работа состоит из введения, двух глав и заключения. В первой главе даётся определение основных терминов и понятий, раскрываются сущность ЭОР: классификация, требования, предъявляемые к ЭОР, инструменты для проектирования и критерии оценки качества ЭОР.
Во втором параграфе в соответствии с выделенными критериями приводится анализ ресурсов, которые используются для преподавания ОРКСЭ (модуль «Основы православной культуры») и других дисциплин, связанных с теологией. Во второй главе приводится проект образовательного ресурса, опираясь на результаты исследования, полученные в первой главе.
Предоставляются критерии оценки результативности созданного ЭОР, обосновывая, почему он обладает большим педагогическим потенциалом и как на основе его можно создать образовательный ресурс.
Курсовая работа опирается на исследования педагогов и методистов, занимающихся проблемами цифровизации и внедрения электронных образовательных ресурсов в образовании (статьи таких авторов, как Ю.М.~Шишкина, Л.Х.~Гаттарова, М.С.~Шевардина, Л.Н.~Тернова, Т.В.~Стебеняева С.Н.~Ларин Е.В.~Герасимова и др.).
