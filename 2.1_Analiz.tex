\section{ЭОР в преподавании теологических дисциплины}

\subsection{Анализ качества имеющихся ресурсов по ОРКСЭ}
\ref{tasks}
Российская электронная школа предоставляет набор образовательных ресурсов по предметам основного общего образования. Хотя образовательные ресурсы по техническим дисциплинам в целом находятся на высоком уровне, в различных источниках имеются указания на слабо проработанные инструменты преподавания социально-гуманитарных дисциплин, где большое внимание уделяется способности ребенка выражать собственное мнение.\cite{13}
В качестве ЭОР для анализа мы выбрали «Основы духовно-нравственной культуры народов России». Учебник включает несколько разделов, мы рассмотрим урок «Христианство. Основные каноны»\cite{14}. Основной материал включает видео лекцию (7 минут 56 секунд) и текст, который дублирует содержимое видеоролика. Урок повествует о Христианстве в культурологическом ключе, как об одной из традиционных мировых религий.
В начале показана религия как феномен деятельности человека, показаны религиозные практики иудаизма, ислама и христианства: видео содержит в одном ряду древние наскальные рисунки, католические статуарные изображения святых, статую Будды, здания и религиозные символы ислама, иудаизма и западного христианства. Важные определения («религия», «мировые религии») продублированы текстом в видеоролике. Среднее время показа картинки 3-9 секунд.
Смысловой ряд понятий логичен и последователен: «религия»; мировые религии; религиозные тексты; христианство; Библия; Христос-Мессия; Рождество; Крещение; Воскресенье Исуса Христа; Крест; Христиане; Бог; Душа; Причастие; молитва. В конце даётся понятие греха, а в качестве иллюстрации – картины Иеронима Босха.
Видеоряд тесно связан с текстом лекции, содержит различные предметы человеческой культуры (фотографии зданий православных храмов и мечетей, православные иконы, католические мозаики и протестантские картинки, изображающие жизнь Иисуса Христа и учеников).
Наша оценка ЭОР. Не выделено главное. Нет тезисов для запоминания. Видеоряд должен создавать межпредметные связи, но гиперссылки на другие смежные ресурсы и подписи на изображениях отсутствуют. Урок дает лишь общее представление о христианской религии. В соответствии с критериями оценки такая лекция обеспечивает лишь пассивный уровень интерактивности, ему не хватает (действий по выбору пользователя). В конце отсутствует тестовый блок и рекомендованные материалы для изучения.
В качестве альтернативы и дополнения можно предложить систему тестирования, предложенная на сайте azbyka.ru. Качество фотовикторин и квизов на на azbyka.ru по критерию интерактивности заметно превосходит качество тестов РЭШ по нашему и другим предметам. Например, фотовикторина представляет четыре варианта ответа: при выборе правильного или неправильного варианта ответа даётся разъяснение. Гиперссылка на внутренний ресурс, где можно прочитать подробнее о предмете, независимо от того правильно или неправильно выбран ответ. Отсутствует реклама, ничего не отвлекает от предмета, и позволяет получить исчерпывающую информацию. На сайте есть кроссворды (созданы при помощи https://www.crossword-compiler.com/). Сайт постоянно пополняется литературой, использует потенциал гипертекстовой разметки – позволяет осуществлять навигацию внутри книги. Личный кабинет сохраняет истории пройдённых ссылок и прочтённых текстов. Такой подход помогает в реализации идеи индивидуальных образовательных траекторий, но всё же это не полноценная LMS.
Проект «Библия для детей»\cite{15}. Первая – книга, которая содержит изложение основных Ветхо- и Новозаветных сюжетов, адаптированных для восприятия современного школьника 10-12 лет. Текст готовится в сотрудничестве с педагогами и психологами, которые ориентированы на школьное образование именно этого возраста. Его сопровождают необходимые пояснения, по которым можно получить дополнительно аудио и видео материалы, произведения искусства, которые предназначены для восприятия в единстве с текстом и помогают раскрыть идеи и образы, заложенные в библии. Проект показывает, что русская культура вдохновлена библией. Из недостатков – нет системы проверки знаний.
Образовательный контент по ОРКСЭ, представленный на сайтах для учителей и в социальных сетях: (РгоШколу.ру, Современный учительский портал, Социальная сеть работников образования, федеральный Банк успешных практик по ОРКСЭ), не соответствуют требованиям интерактивности и самодостаточности, предъявляемые к ЭОР, с трудом может быть использован в создании ЭОР. 
Мы попытались доработать презентацию по теме: «Церковные таинства». Цель урока – дать представление о церковных таинствах посредством презентации. Дети в основном не воцерковленные, поэтому презентация должна быть грамотной с точки зрения соподчинённости элементов структуры, и дизайна. На сайте https://infourok.ru/prezentaciya- na-temu-pravoslavnie-tainstva-1989673.html есть презентация, состоящая из 34-х слайдов. Среди недостатков отмечаем:
отсутствует описание чина таинства крещения, с расшифровкой смысла совершаемых священником действий;
отсутствует описание таинства миропомазания, ссылка на Билию, где основание этого таинства;
в слайде «Таинство покаяния» можно заменить фотографию на нашу (на наш взгляд, недостаточно мотивирована фотография патриарха на 14 слайде);
о таинстве евхаристии не упомянуто, что частицы, погружаемые священником в чашу, символизируют всех стоящих в храме, живых и усопших, а также бесплотных сил (15 слайд). На наш взгляд, для детей важно расшифровать понятие «истинное тело и кровь Христа»;
дизайн слайдов не однотипный, золотые ангелы по углам не мотивированы;
текст на слайдах можно сопроводить голосовым сопровождением.
«Библия онлайн» – Библия в синодальном, современном и других переводах. Симфония, словари, собрание толкований и комментариев. «Библия онлайн» помогает читать и изучать библию, если нет возможности купить специальную литературу (словари, симфонии, переводы). На наш взгляд, не подходит для рассматриваемой возрастной группы.
Приложение «Жития святых» для смартфона интересено тем, что оно создано с целью собрать средства для перевода и издания «Житий святых» для китайцев. Авторы служат на приходе свв. Апп. Петра и Павла в Гонконге. Интересный IT проект, сильная идея, можно каждый день прочитывать краткое житие на смартфоне. Учитель может использовать в качестве домашнего задания. Однако считать данный ресурс можно считать скорее дополнительным, чем полноценным ЭОР.
Православный интернет-курс – проект дистанционного содействия вхождению в веру и практику Православной Церкви. \url{https://azbyka.ru/prav} – решает важную задачу предкрещальной подготовки. Данный ресурс, на наш взгляд, соответствует требованиям ЭОР и содержательно находится в русле решения миссионерских задач.
Вывод. Мы проанализировали образовательные ресурсы по ОРКСЭи выявили нехватку ЭОР которые бы отвечали комплексу требований, предъявляемых к электронным образовательным ресурсам. Другие дисциплины, особенно технические намного лучше обеспечены ЭОР. У социально-гуманитарных дисциплин – особая специфика, в идеале ресурс должен стать предметом для сетевого взаимодействия пользователей, чтобы быть предметом осмысления.

