\section{Педагогический потенциал ЭОР в ОРКСЭ =}
Дидактические трудности ОРКСЭ + Дидактические возможности ЭОР.
\label{section2}

В процессе преподавания ОРКСЭ у педагогов возникают следующие дидактические трудности:
\subsection{Пожелания учителей}

\begin{enumerate}
\item 1. Для реализации культурологического подхода в преподавании ОРКСЭ пособие должно быть как можно более наглядным. Бумажные учебники не могут в полной мере обеспечить соблюдение принципа наглядности, поэтому их дополняют QR кодами, по которым можно получить задание. А в тетради остаются только письменные проверочные задания и тесты.

\item 6.коммуникационные технологии (презентации, видеоролики, ) т.к. иллюстративного материала и информации в учебнике недостаточно, к тому же презентации развивают зрительную память и учебный материал лучше усваивается.

\item 2. Безотметочная система предполагает групповую активность детей на уроке --дискуссионные формы работы. 

\item 4. Просмотр фильмов -- организация дискуссии.

\item 3. Малое количество часов в классе -- подготовка переносится на дом.  7.самостоятельно готовят сообщения к урокам,

\item 2. Отсутствие квалифицированных Преподавателей -- видеоролики, вопросы  родителям.

\item 5. В книге для учителя мало материала, непосредственно направляющего к разработкам уроков. Хотелось  бы иметь под рукой такую книгу, которая помогала бы учителю при составлении конспектов. Приходится использовать различные справочники, энциклопедии, что отнимает очень много времени. Вместе с тем хорошую помощь и поддержку при подготовке к урокам оказывают материалы, разработки  уроков размещенные  на интернет- сайтах. 

\end{enumerate}

Примеры заданий в рабочей тетради:
\begin{itemize}
\item филворд
\item кроссворды
\item найти соответствие между термином и его определением,
\item предлагаю ребятам самим выбрать задание и рассказать о нём все, что знают
\item тесты
\end{itemize}

В качестве практического предложения можно внести небольшое улучшение в работу образовательного портала миссионерского института. А именно: выкладываем лекции только аудио, к ним будет проще получить доступ с телефона. Это делается мгновенно при помощи ffmpeg

Cовременные разработчики сервисов придерживаются концепции mobile first. в т.ч в образовании.