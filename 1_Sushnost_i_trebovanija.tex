\newpage
\large
\section {Электронные образвотельные ресурсы: сущность и требования}
\label{section1}
Для того, чтобы определить сущность понятия Электронный Образовательный Ресурс (ЭОР) необходимо рассмотреть это понятие в ряду близких понятий, таких как: <<образовательный контент>>, <<электронный учебник>>, <<система дистанционного обучения>>, <<цифровая образовательная среда>> и т.д.

Образовательный контент – структурированное предметное содержание. (тексты, изображения, аудиовизуальная продукция и пр.), используемое в. образовательном процессе. 
Цифровой образовательный контент – это материалы и средства обучения и воспитания, представленные в электронном виде, включая информационные ресурсы, в том числе входящие в состав открытой информационно-образовательной среды «Российская электронная школа», а также средства, способствующие определению уровня знаний, умений, навыков, средства оценки компетенций и достижений учащихся, разрабатываемые и (или) предоставляемые поставщиками контента и образовательных сервисов для организации деятельности цифровой образовательной среды.

Электронный учебник --- это электронное издание, которое в систематизированном виде
воспроизводит содержание учебной дисциплины в соответствии с официально утвержденной учебной программой и требованиями дидактики и не может быть трансформировано в печатный аналог без утраты дидактических свойств\cite{balalaeva}. 

Обычно подчёркивается, что электронные учебники не являются
только результатом перевода в электронный формат уже имеющихся учебных
материалов, электронный учебник это не оцифрованная копия бумажного учебника, но образовательный контент, который объединяет средства обучения, практики и контроля знаний\cite[c.53]{blinov}.

Под электронным образовательным ресурсом понимают совокупность учебных, учебно-методических и/или контрольно-измерительных материалов, представленную в виде определенной информационно-технологической конструкции, удобной для изучения и
использования в процессе обучения\cite{jurkina20}. Главная цель использования ЭОР на уроках --- вывести образовательный процесс на новый уровень за счет применения современных инфомационно-коммуникационных технологий. Грамотно сконструированный ЭОР позволяет повысить интерес к обучению и оживить учебный материал за счёт обобщения и систематизации тематических смысловых блоков и визуализации учебного материала, используемого педагогом на уроке.

Как видим, понятия электронный учебник и электронный образовательный ресурс близки по содержанию, однако мы предполагаем, однако что понятие ЭОР несколько шире, и как-то связано с системой дистанционного обучения и цифровой образовательной средой.

Электронная система дистанционного обучения (СДО) — это интернет-платформа, в которой можно дистанционно обучать сотрудников: назначать видеоуроки, книги и курсы, тестировать и следить за успеваемостью. 

Цифровая образовательная среда, как указано в Постановлении Правительства РФ от 7 декабря 2020 г. № 2040 «О проведении эксперимента по внедрению цифровой образовательной среды», – это совокупность условий для реализации образовательных программ начального общего, основного общего и среднего общего образования с применением электронного обучения, дистанционных образовательных технологий с учетом функционирования электронной информационно-образовательной среды, включающей в себя электронные информационные и образовательные ресурсы и сервисы, цифровой образовательный контент, информационные и телекоммуникационные технологии, технологические средства и обеспечивающей освоение учащимися образовательных программ в полном объеме независимо от места их проживания.

\hyperref[task1]{В сущности} ЭОР –  образовательный контент,  представленный в электронно-цифровой форме и включающий в себя структуру, предметное содержание и выходные данные, которые позволяют осуществить его каталогизацию и классификацию и включать его в систему дистанционного образования и цифровую образовательную среду\cite{gost}. Иными словами это законченный модуль по какой либо теме или дисциплине, обладающего свойствами самодостаточности (возможность использования без доступа к интернету) мультимедийности и интерактивности, готовый для включения в цифровую образовательную среду.

\section{ЭОР в преподавании теологических дисциплины}
Педагогический потенциал представляет собой свойство, присущие социально значимым предметам и явлениям, посредством которых становится возможным формирование и становление личности в процессе её образования. Ряд ученых постепенно пришли к консенсусу о том, что для достижения максимального эффекта влияния электронных ресурсов в образовательной среде требуется согласование связей между всеми компонентами технологии (аппаратное и программное обеспечение), образовательным контекстом и пользователями (преподавателями и студентами)\cite{ivinskaya22}.

Для выявления  \hyperref[goal]{педагогического потенциала} ЭОР в преподавании теологических дисциплин необходимо определить место и функцию теологии в общей структуре школьных предметов. В системе образования теология, выполняет важную мировоззренческую и воспитательную функции, имеет тесную связь с литературой, историей, философией и другими предметами, формирующими мировоззрение человека, помогает установить связи между естественными и гуманитарными науками.

Курс ОРКСЭ – составная часть единого образовательного пространства духовно-нравственного развития и воспитания обучающегося, включающего урочную, внеурочную, внешкольную и общественно полезную деятельность.

Цель учебного курса ОРКСЭ – формирование у младшего подростка мотиваций к осознанному нравственному поведению, основанному на знании и уважении культурных и религиозных традиций многонационального народа России, а также к диалогу с представителями других культур и мировоззрений.

%Задачи учебного курса ОРКСЭ:
%\begin{enumerate}
%\item знакомство обучающихся с основами православной, мусульманской, буддийской, иудейской культур, основами мировых религиозных культур и светской этики;
%\item развитие представлений младшего подростка о значении нравственных норм и ценностей для достойной жизни личности, семьи, общества;
%\item обобщение знаний, понятий и представлений о духовной культуре и морали полученных обучающимися в начальной школе, и формирование у них  ценностно-смысловых мировоззренческих основ, обеспечивающих целостное восприятие отечественной истории и культуры при изучении гуманитарных предметов на ступени основной школы;
%\item развитие способностей младших школьников к общению в полиэтнической и многоконфессиональной среде на основе взаимного уважения и диалога во имя общественного мира и согласия.
%\end{enumerate}

В преподавании ОРКСЭ используется культурологический подход, который нацелен на формирование у школьников представления о религии как о важнейшей составляющей мировой культуры.

В парадигме цифровизации образования меняются условия коммуникации, учитель перестаёт быть «транслятором знаний», ведущими направлениями деятельности преподавателя становится сопровождение и поддержка, создание условий для общения.
Современные школьники, используя цифровые ресурсы, учатся искать информацию по поставленным вопросам, совершенствовать свои умения в переработке и предоставлении информации (подготовка рефератов, презентаций, сообщений докладов), между пользователями осуществляется сетевой обмен\cite{shevardina}.
В условиях перехода на смешанное обучение, традиционные аудиторные виды занятий переносятся во внеаудиторную, самостоятельную часть работы (дистант), а учебная деятельность становится метапредметной, поскольку жизненный опыт показывает, что решение любой практической задачи всегда требует привлечения знаний из нескольких предметных областей поэтому необходимо внимательно отнестись к развитию межпредметных связей в образовании\cite{metapredmet}.
Существует три уровня межпредметных связей: 1) внутри предметов одного цикла (например, ботаника -- биология); 2) внутри предметов естественнонаучного цикла (биология и физика, педагогика и психология); 3) показать обучаемым связь частных понятий с философскими категориями.

%Для реализация межпредметного взаимодействия требуется определить функции предмета в общей системе образования.

%Данные исследования помогут обеспечить должную интерактивность учебного процесса и помочь ученикам в организации самообучения и выбора направлений обучения\cite{osobennosti_eor}.

%Таким образом, для успешной интеграции в цифровую образовательную среду образовательный контент должен соответствовать принципам дидактики (быть доступным и наглядным; последовательно и систематично излагаться).
%Модули должны иметь ссылки на естественно-научные и гуманитарные предметы, создавая и поддерживая таким образом межпредметные связи и соответствовать высоким требованиям редакторской правки и верстки текста.

\large
\subsection{Требования, предъявляемые к ЭОР}
Как мы уже упомянули, обязательными характеристиками ЭОР по сравнению с <<традиционными>> учебными средствами являются  являются интерактивность, мультимедийность и самодостаточность: 

Интерактивность --- возможность выполнения действий по выбору пользователя – качественное понятие, отражающее уровень активности пользователя, которое определяется функциональными возможностями учебного продукта.
Исследователи выделяют четыре уровня интерактивности: «простой или пассивный, ограниченный, полноценный и уровень реального масштаба времени»\cite{kochisov15}.
%При минимальном уровне смысловая нагрузка лежит на текстовых компонентах, навигация осуществляется по элементам текста. При максимальном уровне смысловая нагрузка распределяется по визуальным объектам.
Мультимедийность --- использование информации разных видов (видео, звук, графика, базы данных).
Самодостаточность --- наличие всех необходимых для использования ресурса материалов с учетом особенности категории пользователя, отсутствие рекламы.
Кроме того, существует ряд требований, предъявляемых к электронным образовательным ресурсам, которые необходимо учитывать при оценке и проектировании ЭОР:
\begin{itemize}
\item Функциональные требования\cite[c. 85]{larin}. 
\item Технико-технологические требования\cite[c. 87]{larin}. 
\item Дидактические требования\cite[c. 87]{larin}. 
\item Методические требования\cite[c. 89]{larin}. 
\item Требования к дизайну и психологическому восприятию электронного учебника\cite[c. 90]{larin}.
\end{itemize}
\hyperref[goal]{Цель?}

