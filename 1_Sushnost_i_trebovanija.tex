\newpage
\large
\section {Электронные образвотельные ресурсы: сущность и требования}
Для того, чтобы определить сущность понятия Электронный Образовательный Ресурс (ЭОР) необходимо рассмотреть это понятие в ряду близких понятий, таких как: <<образовательный контент>>, <<электронный учебник>>, <<система дистанционного обучения>>, <<цифровая образовательная среда>> и т.д.

Образовательный контент – структурированное предметное содержание. (тексты, изображения, аудиовизуальная продукция и пр.), используемое в. образовательном процессе. 
Цифровой образовательный контент – это материалы и средства обучения и воспитания, представленные в электронном виде, включая информационные ресурсы, в том числе входящие в состав открытой информационно-образовательной среды «Российская электронная школа», а также средства, способствующие определению уровня знаний, умений, навыков, средства оценки компетенций и достижений учащихся, разрабатываемые и (или) предоставляемые поставщиками контента и образовательных сервисов для организации деятельности цифровой образовательной среды.

Электронный учебник --- это электронное издание, которое в систематизированном виде
воспроизводит содержание учебной дисциплины в соответствии с официально утвержденной учебной программой и требованиями дидактики и не может быть трансформировано в печатный аналог без утраты дидактических свойств\cite{balalaeva}. 

Обычно подчёркивается, что электронные учебники являются не
только результатом перевода в электронный формат уже имеющихся учебных
материалов, электронный учебник это не оцифрованная копия бумажного учебника, но образовательный контент, который объединяет средства обучения, практики и контроля знаний.

Под электронным образовательным ресурсом понимают совокупность учебных, учебно-методических и/или контрольно-измерительных материалов, представленную в виде определенной информационно-технологической конструкции, удобной для изучения и
использования в процессе обучения\cite{jurkina20}.

Как видим, понятия электронный учебник и электронный образовательный ресурс близки по содержанию, однако мы предполагаем, однако что понятие ЭОР несколько шире, и стоит в этом ряду ближе к понятию СДО.

Электронная система дистанционного обучения (СДО) — это интернет-платформа, в которой можно дистанционно обучать сотрудников: назначать видеоуроки, книги и курсы, тестировать и следить за успеваемостью. Однако понятие СДО значительно шире, чем понятие единичного электронного образовательного ресурса. Казалось бы --- то же самое: обучение, практика, тест. Однако под электронным учебником понимается что то вроде законченного модуля по какой либо дисциплине или теме, обладающего свойствами самодостаточности (возможность использования без доступа к интернету)

\subsection{Сущность и особенности ЭОР}
ЭОР – образовательный ресурс, представленный в электронно-цифровой форме и включающий в себя структуру, предметное содержание и метаданные о них. ЭОР имеет определенную структуру, предметное содержание и выходные данные, которые позволяют осуществить каталогизацию и классификацию ЭОР\cite{gost}. То есть ЭОР можно рассматривать как элемент цифровой образовательной среды. 

Цифровая образовательная среда, как указано в Постановлении Правительства РФ от 7 декабря 2020 г. № 2040 «О проведении эксперимента по внедрению цифровой образовательной среды», – это совокупность условий для реализации образовательных программ начального общего, основного общего и среднего общего образования с применением электронного обучения, дистанционных образовательных технологий с учетом функционирования электронной информационно-образовательной среды, включающей в себя электронные информационные и образовательные ресурсы и сервисы, цифровой образовательный контент, информационные и телекоммуникационные технологии, технологические средства и обеспечивающей освоение учащимися образовательных программ в полном объеме независимо от места их проживания.

\large
\subsection{Требования, предъявляемые к ЭОР}

Существует ряд требований, предъявляемых к электронным образовательным ресурсам:
\begin{itemize}
\item Функциональные требования. 
\item Технико-технологические требования. 
\item Дидактические требования. 
\item Методические требования. 
\item Требования к дизаину и психологическому восприятию электронного учебника.
\end{itemize}
Обязательными характеристиками ЭОР являются: интерактивность --- возможность выполнения действий по выбору пользователя – качественное понятие, отражающее уровень активности пользователя, которое определяется функциональными возможностями учебного продукта.
Исследователи выделяют четыре уровня интерактивности: «простой или пассивный, ограниченный, полноценный и уровень реального масштаба времени»\cite{kochisov15}.
При минимальном уровне смысловая нагрузка лежит на текстовых компонентах, навигация осуществляется по элементам текста. При максимальном уровне смысловая нагрузка распределяется по визуальным объектам.
Мультимедийность – использование информации разных видов (видео, звук, графика, базы данных).
Самодостаточность – наличие всех необходимых для использования ресурса материалов с учетом особенности категории пользователя, отсутствие рекламы.
Главная цель использования ЭОР на уроках – вывести образовательный процесс на новый уровень за счет применения современных инфомационно-коммуникационных технологий.
Грамотно сконструированный ЭОР позволяет повысить интерес к обучению.
ЭОР также предназначен для:
* визуализации учебного материала, используемого педагогом на уроке;
* обобщения и систематизации тематических смысловых блоков;
* оживления учебного материала.
В новой парадигме меняются условия коммуникации, учитель перестает быть «транслятором знаний», ведущими направлениями деятельности преподавателя становится сопровождение и поддержка, создание условий для общения.
Современные школьники, используя цифровые ресурсы, учатся искать информацию по поставленным вопросам, совершенствовать свои умения в переработке и предоставлении информации (подготовка рефератов, презентаций, сообщений докладов), между пользователями осуществляется сетевой обмен\cite{shevardina}.
В условиях перехода на смешанное обучение, традиционные аудиторные виды занятий переносятся во внеаудиторную, самостоятельную часть работы (дистант), а учебная деятельность становится метапредметной.
Жизненный опыт показывает, что решение любой практической задачи всегда требует привлечения знаний из нескольких предметных областей поэтому необходимо внимательно отнестись к метапредметности и развитию межпредметных связей в образовании\cite{metapredmet}.
Существует три уровня межпредметных связей: 1) внутри предметов одного цикла (например, ботаника -- биология); 2) внутри предметов естественнонаучного цикла (биология и физика, педагогика и психология); 3) показать обучаемым связь частных понятий с философскими категориями.
Для реализация межпредметного взаимодействия требуется определить функции предмета в общей системе образования.
Данные исследования помогут обеспечить должную интерактивность учебного процесса и помочь ученикам в организации самообучения и выбора направлений обучения\cite{osobennosti_eor}.
Таким образом, для успешной интеграции в цифровую образовательную среду образовательный контент должен соответствовать принципам дидактики (быть доступным и наглядным; последовательно и систематично излагаться).
Модули должны иметь ссылки на естественно-научные и гуманитарные предметы, создавая и поддерживая таким образом межпредметные связи и соответствовать высоким требованиям редакторской правки и верстки текста.
Достижение поставленных целей требует большого внимания к проектированию, для того, чтобы продукт оставался востребованным в образовательном процессе, желательно, чтобы процесс проектирования и создания ЭОР опирался на административные регламенты ВУЗов.

Разработка электронного образовательного ресурса должна осуществлялась в соответствии с нормативными документами и базироваться на следующих принципах:
соответствие образовательному стандарту – теоретические и практические учебные материалы направлены на освоение образовательной программы по соответствующей дисциплине, на формирование соответствующих компетенций;
системность – логическая, функциональная связанность учебных материалов;
модульность – учебные материалы представлены в законченных блоках;
полнота и комплектность – в состав электронного образовательного ресурса входят справочный, теоретико-практический, методический и контрольно-диагностический модули;
последовательность изложения материала – логика изложения учебного материала должна быть построена на соблюдении причинноследственных, временных, логических связей;
соответствие объема учебных материалов объему зачетных единиц по дисциплине – учебные материалы должны быть представлены в объеме, соответствующим зачетным единицам;
актуализация – все учебные материалы должны обновляться с определенной периодичностью с учетом новых научных достижений.
Составной частью ЭОР является электронный учебник (учебное пособие).
Перед тем, как приступить к разработке электронного учебника (учебного пособия) необходимо учесть соответствие современным положениям дидактической теории, помнить, что электронное издание отличается от печатного учебного материала: иной интерфейс, мультимедийность, иные средства наглядности – динамические визуальные формулы логически связанные с демонстрацией процессов, интерактивная графика, трёхмерная анимация, минимум текстовых блоков, лаконичные логично расположенные предложения, возможности создания тестовых материалов с применением новых интерактивных возможностей\cite{7}.