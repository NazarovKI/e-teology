\large
\newpage
\large
\subsection{Инструменты для проектирования электронных образовательных ресурсов}
Поскольку мы вяснили, что ЭОР, это прежде всего контент с выходными данными, которые позволяют структурировать его, то не вижу смысла давать здесь анализ програмных продуктов. Об этом можно почитать\cite{mahmutova19, larin}. Хочется лишь отметить ключевые тенденции, как изменились компьютерные программы: 

Если раньше, большая энциклопедия Кирилла и Мефодия распрастранялась на компакт-дисках и имела свои ограничения для воспроизведения (системные требования), решающее значение имело "железо" компьютера и операционная система, то сейчас(?) произошло отделение содержимого от формы его представления. На уровне компьютера контент представляется в виде различных текстовых и бинарных файлов, структурированных в каталоги и воспроизводимых на любой в любой операционной системе. 

Пользователю же, контент предоставляется посредством посредством суммы технологий, позволяющих сделать его доступным в web-браузере ноутбука, планшета или смартфона. Набор продуктов и технологий может отличаться в зависимости от поставляемой задачи. 

Отдельно хочется отметить различия проприетарного и открытого програмного обеспечения.

В настоящее время существуют как готовые инструменты для конструирования ЭОР, так и автоматизированные системы электронного обучения, которые применяют для обучения сотрудников коммерческие организации.
Перечислим, некоторые программные возможности, которые сегодня используются в подготовке ЭОР, например: iSpring Suite, Smart Builder, CourseLab, MOS Solo, Zenler, Easygenerator, Lesson Writer, Гиперметод, Blackboard, LMS Moodle, Media Трансформер.
Это далеко не полный список программных продуктов (платформ) для создания интерактивных учебных материалов, но все их объединяет возможность включать гипертекст, аудио- и видео- материалы. Например, ISpring Suite позволяет работать со слайдами, сделанными в программе PowerPoint: можно связывать звуковое сопровождение, однако её функционал недостаточен, поскольку учебный материал представлен с преимущественно линейной навигацией, ограничены способы создания интерактивных упражнений для самоконтроля и тренинга. Данный продукт является коммерческим. В настоящее время многие отдают предпочтение open source продуктам, как более гибким. Так, М.В.~Махмутова, выбирая инструмент для технического курса, останавливает предпочтение на конструкторе «LMS Moodle» \cite{mahmutova19, larin}. 
Система Moodle распространяется бесплатно в виде набора компонент с открытым исходным кодом по лицензии GNU GPL, что обеспечивает возможность ее использования без привлечения дополнительных финансовых затрат. Эта система представляет собой комплексный программный продукт, на базе которого может быть сформирована единая ИОС, позволяющая обеспечить набор сервисов сетевого обучения, доступ и управление программными инструментами, цифровыми ресурсами, техническими и пользовательскими приложениями, структурированными данными. М.В.~Махмутова, создавая собственный ЭОР, начинает с контрольного среза знаний, затем идёт лекция (автор приводит требования к электронной лекции), в завершении система должна позволять проводить лабораторные и практические работы.
В другом исследовании\cite{dementeva17}, сравниваются популярные коммерческие конструкторы электронных образовательных ресурсов: CourseLab, Easygenerator, Document Suite, eAuthor CBT, iSpring Free, Microsoft LCDS.

Существуют информационные технологии, которые применяются и для разработки электронных образовательных ресурсов: язык разметки НТМL, язык JavaScript: редакторы, технологии взаимодействия сервера с клиентом\cite[С. 154.]{dementeva17}.

Гораздо лучше создавать ресурсы, где всё обучение, практика и контроль знаний 

Разработчики продолжают и сегодня создавать преимущественно текстовый образовательный контент, но дополняют их интерактивными видео- и аудио- контентом\cite{opk}. Недостатками этого ресурса является отсутствие самодостаточности как в плане контента (нет ссылок, некоторого внешнего окружения), так и в том плане, что для воспроизведения файлов нужны внешние программы (word, аудио проигрыватель). 

-доступны в окне браузера.

Пример качественного электронного учебника в технических дисциплинах – ЭОР по устройству автомобиля\cite{automobile}, соответствует требованиям предъявляемым к ЭОР: он интерактивный, есть мультимедийные материалы, самодостаточный (отсутствие рекламы), соблюдены высокие стандарты грамотности текста и верстки. Проект осуществлен при поддержке Представительства зарегистрированного общества «Deutscher Volkshochschul-Verband e.V.» (ФРГ) в Республике Беларусь, и имеет хорошую техническую поддержку.

При создании образовательных ресурсов для школы можно использовать опыт, накопленный при подготовке вузовских ЭОР: использовать платформы, построенные на принципах open source (памятуя об их ограничениях и о предоставляемых возможностях совместной работы над проектами), и  на коммерческих платформах с хорошей технической поддержкой.

