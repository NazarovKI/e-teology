\large
\subsection{Инструменты для проектирования ЭОР}
Этапы разработки электронных образовательных ресурсов в общем виде соответствуют следующей схеме:
* Подготовительный этап: анализ потребности в электронных образовательных ресурсах, определение текущей обеспеченности дисциплины, определение задач электронных образовательных ресурсов.
* Проектирование и разработка: поиск и подбор источников, структуризация учебных материалов, создание текстовых и мультимедийных материалов, разработка контрольно-диагностических материалов.
* Апробация: осуществление образовательного процесса с использованием электронного образовательного ресурса, при необходимости, его корректировка.
* Экспертиза: разработанный электронный образовательный ресурс проходит все виды экспертиз (содержательную, педагогическую, эргономическую и др.).
В настоящее время существуют как готовые инструменты для конструирования ЭОР, так и автоматизированные системы электронного обучения, которые применяют для обучения сотрудников коммерческие организации.
Перечислим, некоторые программные возможности, которые сегодня используются в подготовке ЭОР, например: iSpring Suite, Smart Builder, CourseLab, MOS Solo, Zenler, Easygenerator, Lesson Writer, Гиперметод, Blackboard, LMS Moodle, Media Трансформер.
Это далеко не полный список программных продуктов (платформ) для создания интерактивных учебных материалов, но все их объединяет возможность включать гипертекст, аудио- и видео- материалы.
Например, ISpring Suite позволяет работать со слайдами, сделанными в программе PowerPoint: можно связывать звуковое сопровождение, однако её функционал недостаточен, поскольку учебный материал представлен с преимущественно линейной навигацией, ограничены способы создания интерактивных упражнений для самоконтроля и тренинга.
В мире информационных технологий все быстро меняется, поэтому необходимо выбрать инструмент соответственно обучающим задачам.
Так, М.В.~Махмутова, выбирая инструмент для технического курса, останавливает предпочтение на конструкторе «LMS Moodle» \cite{8}. 
Система Moodle распространяется бесплатно в виде набора компонент с открытым исходным кодом по лицензии GNU GPL, что обеспечивает возможность ее использования без привлечения дополнительных финансовых затрат. Эта система представляет собой комплексный программный продукт, на базе которого может быть сформирована единая ИОС, позволяющая обеспечить набор сервисов сетевого обучения, доступ и управление программными инструментами, цифровыми ресурсами, техническими и пользовательскими приложениями, структурированными данными.
М.В.~Махмутова, создавая собственный ЭОР, начинает с контрольного среза знаний, затем идёт лекция (автор приводит требования к электронной лекции), в завершении система должна позволять проводить лабораторные и практические работы.
В другом исследовании, сравниваются популярные коммерческие конструкторы электронных образовательных ресурсов: CourseLab, Easygenerator, Document Suite, eAuthor CBT, iSpring Free, Microsoft LCDS.
Существуют информационные технологии, которые применяются и для разработки электронных образовательных ресурсов: язык разметки НТМL, язык программирования JavaScript, графические редакторы, технологии взаимодействия сервера с клиентом\cite{10}.
Разработчики продолжают и сегодня создавать преимущественно текстовые образовательные ресурсы при помощи простых технологий (язык разметки HTML, язык программирования JavaScript)\cite{11}, но дополняют их интерактивными видео- и аудио- ресурсами.
Пример качественного электронного учебника в технических дисциплинах – ЭОР по устройству автомобиля\cite{12}, соответствует требованиям предъявляемым к ЭОР: он интерактивный, есть мультимедийные материалы, самодостаточный (отсутствие рекламы), соблюдены высокие стандарты грамотности текста и верстки.
Проект осуществлен при поддержке Представительства зарегистрированного общества «Deutscher Volkshochschul-Verband e.V.» (ФРГ) в Республике Беларусь, и имеет хорошую техническую поддержку.
При создании образовательных ресурсов для школы можно использовать опыт, накопленный при подготовке вузовских ЭОР: использовать платформы, построенные на принципах open source (памятуя об их ограничениях и о предоставляемых возможностях совместной работы над проектами), и  на коммерческих платформах с хорошей технической поддержкой.

