\large
\newpage
\subsection{Инструменты для проектирования электронных образовательных ресурсов}
Поскольку мы вяснили, что ЭОР, это прежде всего контент с выходными данными, которые позволяют структурировать его, то не вижу смысла давать здесь анализ програмных продуктов. Об этом можно почитать\cite{mahmutova19, larin}. Хочется лишь отметить ключевые тенденции, как изменились компьютерные программы: 

Если раньше, большая энциклопедия Кирилла и Мефодия распрастранялась на компакт-дисках и имела свои ограничения для воспроизведения (системные требования). Cейчас оборудование и программы позволяют отделить <<контент>> от формы его представления. На уровне компьютера <<контент>> всегда представляется в виде текстовых и мультимедийных файлов, структурированных в каталоги и воспроизводимых в любой операционной системе. 

Конечному пользователю, тот же контент предоставляется в упорядоченном виде посредством 
%посредством суммы технологий, позволяющих сделать его доступным в 
web-браузера ноутбука, планшета или смартфона. Набор продуктов и технологий может отличаться в зависимости от поставляемой задачи. 

Отдельно хочется написать про различия проприетарного и открытого програмного обеспечения.

\subsection{Системы организации предоставления контента}
В настоящее время существуют как готовые инструменты для конструирования ЭОР, так и автоматизированные системы электронного обучения, которые применяют для обучения сотрудников коммерческие организации.
Перечислим, некоторые программные возможности, которые сегодня используются в подготовке ЭОР, например: iSpring Suite, Smart Builder, CourseLab, MOS Solo, Zenler, Easygenerator, Lesson Writer, Гиперметод, Blackboard, LMS Moodle, Media Трансформер.
Это далеко не полный список программных продуктов (платформ) для создания интерактивных учебных материалов, но все их объединяет возможность включать гипертекст, аудио- и видео- материалы. Например, ISpring Suite позволяет работать со слайдами, сделанными в программе PowerPoint: можно связывать звуковое сопровождение, однако её функционал недостаточен, поскольку учебный материал представлен с преимущественно линейной навигацией, ограничены способы создания интерактивных упражнений для самоконтроля и тренинга. Данный продукт является коммерческим. В настоящее время многие отдают предпочтение open source продуктам, как более гибким. Так, М.В.~Махмутова, выбирая инструмент для технического курса, останавливает предпочтение на конструкторе «LMS Moodle» \cite{mahmutova19, larin}. 
Система Moodle распространяется бесплатно в виде набора компонент с открытым исходным кодом по лицензии GNU GPL, что обеспечивает возможность ее использования без привлечения дополнительных финансовых затрат. Эта система представляет собой комплексный программный продукт, на базе которого может быть сформирована единая ИОС, позволяющая обеспечить набор сервисов сетевого обучения, доступ и управление программными инструментами, цифровыми ресурсами, техническими и пользовательскими приложениями, структурированными данными. М.В.~Махмутова, создавая собственный ЭОР, начинает с контрольного среза знаний, затем идёт лекция (автор приводит требования к электронной лекции), в завершении система должна позволять проводить лабораторные и практические работы.
В другом исследовании\cite{dementeva17}, сравниваются популярные коммерческие конструкторы электронных образовательных ресурсов: CourseLab, Easygenerator, Document Suite, eAuthor CBT, iSpring Free, Microsoft LCDS.

Существуют информационные технологии, которые применяются и для разработки электронных образовательных ресурсов: язык разметки НТМL, язык JavaScript: редакторы, технологии взаимодействия сервера с клиентом\cite[С. 154.]{dementeva17}.

