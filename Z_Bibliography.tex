\newpage
\large
\begin{thebibliography}{99}
\bibitem{fsd}  Постановление Правительства РФ от 7 декабря 2020 г. № 2040 «О проведении эксперимента по внедрению цифровой образовательной среды». url: \url{https://www.garant.ru/products/ipo/prime/doc/74922819/} (дата обращения: 25.11.2021).
\bibitem{opk}  Методическое обеспечение экспериментальных уроков по Основам православной культуры для 4-5 классов url: \url{http://experiment-opk.pravolimp.ru/lessons/7} (дата обращения 24.01.2022).
\bibitem{balalaeva} Балалаева Елена Юрьевна Анализ сущности понятия «Электронный учебник» // Вестник Марийского государственного университета. 2016. №4 (24). URL: \url{https://cyberleninka.ru/article/n/analiz-suschnosti-ponyatiya-elektronnyy-uchebnik} (дата обращения: 09.03.2022).
\bibitem{jurkina20} Журкина Мария Ивановна Различные подходы к определению понятия «Электронный образовательный ресурс» // Проблемы педагогики. 2020. №3 (48). URL: \url {https://cyberleninka.ru/article/n/razlichnye-podhody-k-opredeleniyu-ponyatiya-elektronnyy-obrazovatelnyy-resurs} (дата обращения: 09.03.2022).
\bibitem{automobile}  Устройство автомобиля url: \url{http://service.college-ripo.by/wp-admin/123/eER.html} (дата обращения 24.01.2022)
\bibitem{pravobraz} Глава Синодального отдела митрополит Евгений передал новые учебные пособия по основам православной культуры в школы Среднего Урала url: \url{https://pravobraz.ru/lava-sinodalnogo-otdela-mitropolit-evgenij-peredal-novye-uchebnye-posobiya-po-osnovam-pravoslavnoj-kultury-v-shkoly-srednego-urala/}
\bibitem{lastochkin19}  Ласточкин А. В. Цифровая образовательная среда. Региональный опыт построения цифровой образовательной среды в школе / А. В. Ласточкин, М. Ю. Ходов // Региональное образование: современные тенденции. – 2019. – № 3(39). – С. 45-48.
\bibitem{sample}  Возникновение и распространение христианства url: \url{https://resh.edu.ru/subject/lesson/436/} (дата обращения 25.01.22.
\bibitem{biblechild}  Библия для детей. url: \url{https://bibleforchildren.ru/about-project.html} (дата обращения 22.01.2022)
\bibitem{gost}  ГОСТ Р 52653-2006, статья 12, подраздел 3.2 // Электронный фонд правовой и нормативно-технической документации. 2008 url: \url{http://docs.cntd.ru/document/1200053103} (дата обращения: 06.12.2021).
\bibitem{kochisov15}  Кочисов В. К. Электронный образовательный ресурс как новый педагогический инструмент в условиях развития межпредметных связей //Образовательные технологии и общество. 2015. Т. 18. № 4. С. 615.
\bibitem{shevardina}  Шевардина М.С. Интенсификация учебного взаимодействия в процессе дистанционного обучения теологии // Современные проблемы науки и образования. – 2014. – № 1; url: \url{https://science-education.ru/ru/article/view?id=12127} (дата обращения: 25.01.2022).
\bibitem{metapredmet}  Семенова Н. Г., Томина И. П. Межпредметный метод проектов в условиях комплексного использования электронных образовательных ресурсов //Вестник Оренбургского государственного университета. – 2017. – №. 10 (210).
\bibitem{osobennosti_eor}  Шишкина  Ю. М. Особенности электронного образовательного ресурса. Сборник статей Международной научно-практической конференции, Магнитогорск: "ОМЕГА САЙНС 2020. С. 180-184.
\bibitem{dementeva17}  Дементьева Ю. В. Электронный учебник как основной образовательный ресурс учебного обеспечения электронного обучения / Ю. В. Дементьева // Современные образовательные технологии : монография. – Новосибирск : Общество с ограниченной ответственностью "Центр развития научного сотрудничества", 2017. – С. 137-145.
\bibitem{mahmutova19}  Махмутова, М. В. Технология разработки и применения электронных образовательных ресурсов в учебном процессе вуза / М. В. Махмутова, Е. И. Сеничева, О. А. Акимова // Открытое образование. – 2019. – Т. 23. – № 6. – С. 50-58. – DOI 10.21686/1818-4243-2019-6-50-58.
\bibitem{larin} Ларин С. Н., Герасимова Е. В. Критерии оценки эффективности образовательного процесса на основе применения современных ИКТ // Актуальные вопросы современной науки. 2012. №24. URL: \url{https://cyberleninka.ru/article/n/kriterii-otsenki-effektivnosti-obrazovatelnogo-protsessa-na-osnove-primeneniya-sovremennyh-ikt} (дата обращения: 09.03.2022).
\bibitem{blinov} Блинов В. И. Проект дидактической концепции цифрового профессионального образования и обучения / В. И. Блинов, М. В. Дулинов, Е. Ю. Есенина, И. С. Сергеев /М.: Издательство Перо, 2019 [Электронный ресурс]. URL: \url{https://firo.ranepa.ru/files/docs/proect_didacticheskoy_concepcii_cifrovogo_prof_obr.pdf} (дата обращения: 13.05.2020).
\bibitem{ivinskaya22} Ивинская Марина Сергеевна Педагогический потенциал электронных ресурсов и электронной образовательной среды в вузе // БГЖ. 2018. №3 (24). URL: \url{https://cyberleninka.ru/article/n/pedagogicheskiy-potentsial-elektronnyh-resursov-i-elektronnoy-obrazovatelnoy-sredy-v-vuze} (дата обращения: 09.03.2022).
\bibitem{hse_comunication} Добрякова М. С.,  Юрченко О. В. ДИСТАНЦИОННОЕ ОБУЧЕНИЕ В ШКОЛЕ: ТРИ УРОВНЯ КОММУНИКАТИВНЫХ ТРУДНОСТЕЙ. Выпуск No 14, © НИУ «Высшая школа экономики», 2021. \url{https://www.hse.ru/data/2021/04/09/1391475392/release_14_2021.pdf} (дата обращения: 09.03.2022).
%\bibitem{9}  Best Open Source LMS for Creating Online Course Websites url: \url{https://itsfoss.com/best-open-source- lms/} (дата обращения 24.01.22).
\end{thebibliography}
\end{document}