\newpage
\subsection{Вывод по первой главе.}\todo{Разобраться с форматами ссылок. Нужна ссылка на задачи.}
В \hyperref[task1]{результате} исследования мы выяснили, что в сущности электронный образовательный ресурс это... , особенности этого понятия (акциденции) под этим может пониматься и портал, и система управления обучением learning management system, или же наконец -- электронный учебник, который отличается от обычного рядом признаков и требований. 
%Мы рассмотрели ЭОР технических дисциплин и выявили, что в настоящее время ЭОР должен быть доступен в браузере, независимо от платформы.
В процессе создания ЭОР на первом этапе нужно говорить не столько о средствах и инструментах, сколько о концепции и содержании.
В идеале ЭОР должен помогать школьнику и студенту представить объем знаний и компетенции, которые нужно усвоить, должен помогать формировать учебные цели.
Для понимания, какой именно инструмент использовать для создания ЭОР по ОРКСЭ: html, css, js, или готовые конструкторы, нужно проанализировать специфику преподаваемой дисциплины, целевую аудиторию, уже созданные информационные ресурсы по смежным дисциплинам, и возможные затраты.
Важно учитывать: категорию пользователя, решаемые образовательные задачи, актуальность темы, соответствие ФГОС, СанПиН, возможность интеграции в ЦОС.
\listoftodos