\newpage
\large
\section {Электронные образвотельные ресурсы: сущность и требования}
Для того, чтобы определить сущность понятия Электронный Образовательный Ресурс (ЭОР) необходимо рассмотреть это понятие в ряду близких понятий, таких как: <<образовательный контент>>, <<электронный учебник>>, <<система дистанционного обучения>>, <<цифровая образовательная среда>> и т.д.

Образовательный контент – структурированное предметное содержание. (тексты, изображения, аудиовизуальная продукция и пр.), используемое в. образовательном процессе. 
Цифровой образовательный контент – это материалы и средства обучения и воспитания, представленные в электронном виде, включая информационные ресурсы, в том числе входящие в состав открытой информационно-образовательной среды «Российская электронная школа», а также средства, способствующие определению уровня знаний, умений, навыков, средства оценки компетенций и достижений учащихся, разрабатываемые и (или) предоставляемые поставщиками контента и образовательных сервисов для организации деятельности цифровой образовательной среды.

Электронный учебник --- это электронное издание, которое в систематизированном виде
воспроизводит содержание учебной дисциплины в соответствии с официально утвержденной учебной программой и требованиями дидактики и не может быть трансформировано в печатный аналог без утраты дидактических свойств\cite{balalaeva}. 

Обычно подчёркивается, что электронные учебники являются не
только результатом перевода в электронный формат уже имеющихся учебных
материалов, электронный учебник это не оцифрованная копия бумажного учебника, но образовательный контент, который объединяет средства обучения, практики и контроля знаний.

Под электронным образовательным ресурсом понимают совокупность учебных, учебно-методических и/или контрольно-измерительных материалов, представленную в виде определенной информационно-технологической конструкции, удобной для изучения и
использования в процессе обучения\cite{jurkina20}.

Как видим, понятия электронный учебник и электронный образовательный ресурс близки по содержанию, однако мы предполагаем, однако что понятие ЭОР несколько шире, и стоит в этом ряду ближе к понятию СДО.

Электронная система дистанционного обучения (СДО) — это интернет-платформа, в которой можно дистанционно обучать сотрудников: назначать видеоуроки, книги и курсы, тестировать и следить за успеваемостью. Однако понятие СДО значительно шире, чем понятие единичного электронного образовательного ресурса. Казалось бы --- то же самое: обучение, практика, тест. Однако под электронным учебником понимается что то вроде законченного модуля по какой либо дисциплине или теме, обладающего свойствами самодостаточности (возможность использования без доступа к интернету)

\subsection{Сущность и особенности ЭОР}
ЭОР – образовательный ресурс, представленный в электронно-цифровой форме и включающий в себя структуру, предметное содержание и метаданные о них. ЭОР имеет определенную структуру, предметное содержание и выходные данные, которые позволяют осуществить каталогизацию и классификацию ЭОР\cite{2}. То есть ЭОР можно рассматривать как элемент цифровой образовательной среды. 

Цифровая образовательная среда, как указано в Постановлении Правительства РФ от 7 декабря 2020 г. № 2040 «О проведении эксперимента по внедрению цифровой образовательной среды», – это совокупность условий для реализации образовательных программ начального общего, основного общего и среднего общего образования с применением электронного обучения, дистанционных образовательных технологий с учетом функционирования электронной информационно-образовательной среды, включающей в себя электронные информационные и образовательные ресурсы и сервисы, цифровой образовательный контент, информационные и телекоммуникационные технологии, технологические средства и обеспечивающей освоение учащимися образовательных программ в полном объеме независимо от места их проживания.
