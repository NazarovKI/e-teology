\newpage
\large
\section {ЭОР: сущность и требования}
\label{section1}
Для того, чтобы определить сущность понятия Электронный Образовательный Ресурс (ЭОР) необходимо рассмотреть это понятие в ряду близких понятий: <<образовательный контент>>, <<электронный учебник>>, <<система дистанционного обучения>>, <<цифровая образовательная среда>>.

Образовательный контент – структурированное предметное содержание (тексты, изображения, аудиовизуальная продукция и пр.), используемое в образовательном процессе. 

Цифровой образовательный контент – это материалы и средства обучения и воспитания, представленные в электронном виде, включая информационные ресурсы, в том числе входящие в состав открытой информационно-образовательной среды «Российская электронная школа», а также средства, способствующие определению уровня знаний, умений, навыков, средства оценки компетенций и достижений учащихся, разрабатываемые и (или) предоставляемые поставщиками контента и образовательных сервисов для организации деятельности цифровой образовательной среды.

Электронный учебник --- это электронное издание, которое в систематизированном виде
воспроизводит содержание учебной дисциплины в соответствии с официально утвержденной учебной программой и требованиями дидактики и не может быть трансформировано в печатный аналог без утраты дидактических свойств\cite{balalaeva}. 

Обычно подчёркивается, что электронные учебники не являются только результатом перевода в электронный формат уже имеющихся учебных материалов, электронный учебник это не оцифрованная копия бумажного учебника, но специальное электронное издание, которое объединяет средства обучения, практики и контроля знаний\cite[c.53]{blinov}.

Под электронным образовательным ресурсом понимают совокупность учебных, учебно-методических и/или контрольно-измерительных материалов, представленную в виде определенной информационно-технологической конструкции, удобной для изучения и использования в процессе обучения\cite{jurkina20}. 

Главная цель использования ЭОР на уроках --- вывести образовательный процесс на новый уровень за счет применения современных инфомационно-коммуникационных технологий. Грамотно сконструированный ЭОР позволяет повысить интерес к обучению и оживить учебный материал за счёт обобщения и систематизации тематических смысловых блоков и визуализации учебного материала, используемого педагогом на уроке.

Как видим, понятия электронный учебник и электронный образовательный ресурс близки по содержанию, однако понятие ЭОР несколько шире поскольку связано с системой дистанционного обучения и цифровой образовательной средой.

Электронная система дистанционного обучения (СДО) --- это интернет-платформа, в которой можно дистанционно обучать сотрудников: назначать видеоуроки, книги и курсы, тестировать и следить за успеваемостью. 

Цифровая образовательная среда, как указано в Постановлении Правительства РФ от 7 декабря 2020 г. № 2040 «О проведении эксперимента по внедрению цифровой образовательной среды», --- это совокупность условий для реализации образовательных программ начального общего, основного общего и среднего общего образования с применением электронного обучения, дистанционных образовательных технологий с учётом функционирования электронной информационно-образовательной среды, включающей в себя электронные информационные и образовательные ресурсы и сервисы, цифровой образовательный контент, информационные и телекоммуникационные технологии, технологические средства и обеспечивающей освоение учащимися образовательных программ в полном объёме независимо от места их проживания.

\hyperref[task1]{ЭОР}, по сути ---  электронный учебник,  представленный в электронно-цифровой форме и включающий в себя структуру, предметное содержание и выходные данные, которые позволяют осуществить его каталогизацию и классификацию и включать его в систему дистанционного образования и цифровую образовательную среду\cite{gost}. Иными словами это законченный модуль по какой либо теме или дисциплине, обладающего свойствами самодостаточности (возможность использования без доступа к интернету) мультимедийности и интерактивности, готовый для включения в цифровую образовательную среду.


\subsection{Требования, предъявляемые к ЭОР}
Как мы уже упомянули, обязательными характеристиками ЭОР по сравнению с <<традиционными>> учебными средствами являются  являются интерактивность, мультимедийность и самодостаточность: 

\textbf{Интерактивность} --- возможность выполнения действий по выбору пользователя – качественное понятие, отражающее уровень активности пользователя, которое определяется функциональными возможностями учебного продукта.
Исследователи выделяют четыре уровня интерактивности: «простой или пассивный, ограниченный, полноценный и уровень реального масштаба времени»\cite{kochisov15}.
%При минимальном уровне смысловая нагрузка лежит на текстовых компонентах, навигация осуществляется по элементам текста. При максимальном уровне смысловая нагрузка распределяется по визуальным объектам.
\textbf{Мультимедийность} --- использование информации разных видов (видео, звук, графика, базы данных).
\textbf{Самодостаточность} --- наличие всех необходимых для использования ресурса материалов с учетом особенности категории пользователя, отсутствие рекламы.
Кроме того, существует ряд требований\cite[c. 85-90]{larin}, предъявляемых к электронным образовательным ресурсам, которые необходимо учитывать при оценке и проектировании ЭОР, среди которых:
\begin{itemize}
\item Функциональные ;
\item Технико-технологические;
\item Дидактические;
\item Методические;
\item Требования к дизайну и психологическому восприятию электронного учебника.
\end{itemize}

Поскольку \hyperref[goal]{Цель} нашей работы ---  Мы постараемся на основе данных требований выработать критерии оценки существующих электронных образовательных ресурсов.

