\subsection{Образовательные задачи по теологии}

Какие существуют запросы от людей?
Поскольку создание ЭОР требует больших затрат времени труда авторов и программистов, необходимо вначале определить целесообразность создания ЭОР. В беседе с преподавателем воскресной школы выяснили, что для младших школьников целесообразно создать такие ресурсы по следующим темам:
Иконы пресвятой Богородицы явленные в России (в ХХ веке); – есть идея создать интегрированный урок который связывает области знаний Истории и ОПК.
Двунадесятые праздники;
Недели великого поста;
Страстная седмица;
Божественная литургия – желательный вид материала – короткие видео с пояснением священника, по ходу каждого этапа Божественной литургии.
Богослужебная утварь – для раскрытия темы достаточно фотографий с лаконичными текстовыми пояснениями.
ЭОР по дисциплине «Теория и история христианского искусства».


\subsection{Электронные учебники}
Еще одна проблема -- современные электронные учебники. 



Оцифрованный учебник представляет собой традиционный учебный текст, переведённый в электронный вид, снабжённый перекрестными гиперссылками и ссылками на внешние ре-
сурсы, анимацией, возможно видеофрагментами. Его использование, может быть, позволяет создать несколько более высокую учебную мотивацию у совре-
менных детей, нежели обычный образовательный процесс, од-
нако эта мотивация носит внешний и краткосрочный характер.
Использование электронного учебника вместо печатной книги
существенно повышает нагрузку на зрение. Но главная пробле-
ма «оцифрованной» дидактической практики в данном случае
состоит в том, что стратегия учебной деятельности в работе с
учебником не меняется, либо меняется к худшему, теряя свою
гуманистическую составляющую. Это последнее происходит в
силу того, что педагог, доверяя возможностям оцифрованного
учебника, всё больше самоустраняется из образовательного про-
цесса, освобождая место для диалога «учащийся – компьютер».
Даже возможности индивидуализации обучения, которые
несёт с собой оцифровка традиционного образовательного со-
держания, нередко сказываются отрицательно на процессе раз-
вития: ученик замыкается в своей персональной компьютери-
зованной учебной среде, в ущерб групповым формам работы.

То, что предлагаются на правительственном сайте, к примеру \url{https://edu.gov.ru/distance} это недооцифрованные бумажные учебники, в них нет гиперссылок, не использованы все возможности формата pdf, которые придают электронному учебнику целостность, интерактивность и самодостаточность.


В настоящее время выявляется требование, что учебник должен быть доступен из браузера, без использования дополнительного программного обеспечения, только простые технологии (html, css, javascript). 

Для создания целостных учебников не обязательно проводить исследования, это может быть задача структурировать существующую информацию, которая в разных видах (форматах) представленна в сети. Здесь поможет библиотека pandoc \url{https://pandoc.org/} которая позволяет преобразовать документы из одного формата в другой.
\hyperref[task2]{Задачи второй главы...}