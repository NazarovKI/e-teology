\subsection{Образовательные задачи по теологии}

Какие существуют запросы от людей?
Поскольку создание ЭОР требует больших затрат времени труда авторов и программистов, необходимо вначале определить целесообразность создания ЭОР. В беседе с преподавателем воскресной школы выяснили, что для младших школьников целесообразно создать такие ресурсы по следующим темам:
Иконы пресвятой Богородицы явленные в России (в ХХ веке); – есть идея создать интегрированный урок который связывает области знаний Истории и ОПК.
Двунадесятые праздники;
Недели великого поста;
Страстная седмица;
Божественная литургия – желательный вид материала – короткие видео с пояснением священника, по ходу каждого этапа Божественной литургии.
Богослужебная утварь – для раскрытия темы достаточно фотографий с лаконичными текстовыми пояснениями.
ЭОР по дисциплине «Теория и история христианского искусства».

Еще одна проблема -- электронные учебники. Те, что предлагаются на сайте \url{https://edu.gov.ru/distance} ничем не отличаются от бумажных, то есть они имеют структуру -- внутритекстовые ссылки, оглавление, которыми удобно пользоваться, когда держишь в руках бумажную копию книги, но не на экране, когда большая часть привычных для книги свойств -- физически ощущаемы страницы, одновременность их, они сразу все в руках -- отстутсвует. Страницы нужно вызывать, вводить команду, или долго листать.

Если \todo{посмотреть публикации, посвященные цифровой дидактике}, то увидите, что все об этом и говорят. Решений пока немного

 К счастью в pdf -- самый распрастраненный формат для электронных книг можно делать Ссылки на формулы, рисунки, таблицы, главы и т.д. выделяются шрифтом и цветом (и, естественно, должна быть гиперссылкой). Из Интернета в научные тексты переберутся спецсимволы. Например, смайлики — то есть символы, выражающие эмоции. Электронная версия книги может содержать картинки с анимацией. Причём это далеко не фантастика: эта возможность уже сейчас есть в том же TikZ \url{https://habr.com/ru/company/ruvds/blog/518302/}. Adobe Acrobat 3D - программа для добавления 3D обьектов CAD формата в обычный PDF файл. Вы сможете вставлять в файлы формата PDF разнообразные трехмерные объекты в CAD-форматах. Эти самые 3D-объекты возможно покрутить, рассматривая объект под разными углами, увеличивать/уменьшать и т. д.\url{https://filebox.ru/p/adobe-acrobat-3d/}

Сама программа нужна для того чтобы вставлять трёхмерные объекты, а просматривать такие файлы можно обычным Adobe Acrobat седьмой версии!

Можно написать о преимуществах и недостатков html и pdf формата, я считаю лично, что учебник должен быть в pdf, это придаёт ему целостность.

Для создания целостных учебников не обязательно проводить исследования, это может быть задача структурировать существующую информацию, которая в разных видах (форматах) представленна в сети. Здесь поможет библиотека pandoc \url{https://pandoc.org/} которая позволяет преобразовать документы из одного формата в другой.
\hyperref[task2]{Задачи второй главы...}