\section{Проектирование ЭОР}
Достижение поставленных целей требует большого внимания к проектированию, для того, чтобы продукт оставался востребованным в образовательном процессе, желательно, чтобы процесс проектирования и создания ЭОР опирался на административные регламенты ВУЗов\cite{shevardina}.

Разработка электронного образовательного ресурса должна осуществлялась в соответствии с нормативными документами и базироваться на следующих принципах:
\begin{enumerate}
\item соответствие образовательному стандарту – теоретические и практические учебные материалы направлены на освоение образовательной программы по соответствующей дисциплине, на формирование соответствующих компетенций;
\item системность – логическая, функциональная связанность учебных материалов;
\item модульность – учебные материалы представлены в законченных блоках;
\item полнота и комплектность – в состав электронного образовательного ресурса входят справочный, теоретико-практический, методический и контрольно-диагностический модули;
\item последовательность изложения материала – логика изложения учебного материала должна быть построена на соблюдении причинноследственных, временных, логических связей;
\item соответствие объема учебных материалов объему зачетных единиц по дисциплине – учебные материалы должны быть представлены в объеме, соответствующим зачетным единицам;
\item актуализация – все учебные материалы должны обновляться с определенной периодичностью с учетом новых научных достижений.
\end{enumerate}

Этапы разработки электронных образовательных ресурсов в общем виде соответствуют следующей схеме:
\begin{itemize}
\item Подготовительный этап: анализ потребности в электронных образовательных ресурсах, определение текущей обеспеченности дисциплины, определение задач электронных образовательных ресурсов.
\item Проектирование и разработка: поиск и подбор источников, структуризация учебных материалов, создание текстовых и мультимедийных материалов, разработка контрольно-диагностических материалов.
\item Апробация: осуществление образовательного процесса с использованием электронного образовательного ресурса, при необходимости, его корректировка.
\item Экспертиза: разработанный электронный образовательный ресурс проходит все виды экспертиз (содержательную, педагогическую, эргономическую и др.).
\end{itemize}