\subsection{Электронные учебники}

Составной частью ЭОР является электронный учебник (учебное пособие). 
Перед тем, как приступить к разработке электронного учебника (учебного пособия) необходимо учесть соответствие современным положениям дидактической теории, помнить, что электронное издание отличается от печатного учебного материала: иной интерфейс, мультимедийность, иные средства наглядности – динамические визуальные формулы логически связанные с демонстрацией процессов, интерактивная графика, трёхмерная анимация, минимум текстовых блоков, лаконичные логично расположенные предложения, возможности создания тестовых материалов с применением новых интерактивных возможностей\cite{shevardina}.
Cовременные электронные учебники оставляют желать лучшего: оцифрованный учебник представляет собой традиционный учебный текст, переведённый в электронный вид, снабжённый перекрестными гиперссылками и ссылками на внешние ресурсы, анимацией, возможно видеофрагментами. То, что предлагаются на правительственном сайте, к примеру \url{https://edu.gov.ru/distance} это недооцифрованные бумажные учебники, в них нет гиперссылок, не использованы все возможности формата pdf, которые придают электронному учебнику целостность, интерактивность и самодостаточность. 
Есть мнение, требование, что учебник должен быть доступен из браузера, без использования дополнительного программного обеспечения, только простые технологии (html, css, javascript)\cite{hse_comunication}, поскольку при использовании более сложных технологий неизбежно возникает проблема с техническим обеспечением у конечного пользователя. 



Использование ЭУ, может быть, позволяет создать несколько более высокую учебную мотивацию у современных детей, нежели обычный образовательный процесс, однако эта мотивация носит внешний и краткосрочный характер.
Использование электронного учебника вместо печатной книги существенно повышает нагрузку на зрение. Но главная проблема «оцифрованной» дидактической практики в данном случае состоит в том, что стратегия учебной деятельности в работе с учебником не меняется, либо меняется к худшему, теряя свою гуманистическую составляющую. Это последнее происходит в
силу того, что педагог, доверяя возможностям оцифрованного учебника, всё больше самоустраняется из образовательного процесса, освобождая место для диалога «учащийся – компьютер».
Даже возможности индивидуализации обучения, которые несёт с собой оцифровка традиционного образовательного содержания, нередко сказываются отрицательно на процессе развития: ученик замыкается в своей персональной компьютеризованной учебной среде, в ущерб групповым формам работы.

\section{Проектирование ЭОР}
Достижение поставленных целей требует большого внимания к проектированию, для того, чтобы продукт оставался востребованным в образовательном процессе, желательно, чтобы процесс проектирования и создания ЭОР опирался на административные регламенты ВУЗов\cite{shevardina}.

Разработка электронного образовательного ресурса должна осуществлялась в соответствии с нормативными документами и базироваться на следующих принципах:
\begin{enumerate}
\item соответствие образовательному стандарту – теоретические и практические учебные материалы направлены на освоение образовательной программы по соответствующей дисциплине, на формирование соответствующих компетенций;
\item системность – логическая, функциональная связанность учебных материалов;
\item модульность – учебные материалы представлены в законченных блоках;
\item полнота и комплектность – в состав электронного образовательного ресурса входят справочный, теоретико-практический, методический и контрольно-диагностический модули;
\item последовательность изложения материала – логика изложения учебного материала должна быть построена на соблюдении причинноследственных, временных, логических связей;
\item соответствие объема учебных материалов объему зачетных единиц по дисциплине – учебные материалы должны быть представлены в объеме, соответствующим зачетным единицам;
\item актуализация – все учебные материалы должны обновляться с определенной периодичностью с учетом новых научных достижений.
\end{enumerate}

Этапы разработки электронных образовательных ресурсов в общем виде соответствуют следующей схеме:
\begin{itemize}
\item Подготовительный этап: анализ потребности в электронных образовательных ресурсах, определение текущей обеспеченности дисциплины, определение задач электронных образовательных ресурсов.
\item Проектирование и разработка: поиск и подбор источников, структуризация учебных материалов, создание текстовых и мультимедийных материалов, разработка контрольно-диагностических материалов.
\item Апробация: осуществление образовательного процесса с использованием электронного образовательного ресурса, при необходимости, его корректировка.
\item Экспертиза: разработанный электронный образовательный ресурс проходит все виды экспертиз (содержательную, педагогическую, эргономическую и др.).
\end{itemize}
\subsection{Образовательные задачи по теологии}


Какие существуют запросы от людей?
Поскольку создание ЭОР требует больших затрат времени труда авторов и программистов, необходимо вначале определить целесообразность создания ЭОР. В беседе с преподавателем воскресной школы выяснили, что для младших школьников целесообразно создать такие ресурсы по следующим темам:
Иконы пресвятой Богородицы явленные в России (в ХХ веке); – есть идея создать интегрированный урок который связывает области знаний Истории и ОПК.
Двунадесятые праздники;
Недели великого поста;
Страстная седмица;
Божественная литургия – желательный вид материала – короткие видео с пояснением священника, по ходу каждого этапа Божественной литургии.
Богослужебная утварь – для раскрытия темы достаточно фотографий с лаконичными текстовыми пояснениями.
ЭОР по дисциплине «Теория и история христианского искусства».


