3. Дидактические требования.
Образовательный контент, входящий в состав ЭОР, должен отвечать стандартным дидактическим требованиям, соответствующим специфическим закономерностям обучения и, соответственно, дидактическим принципам обучения [7, с. 186]. Рассмотрим их более подробно.
3.1. Научность обучения предполагает учет последних научных достижений, а так же достаточную корректность и научную достоверность образовательного контента, представленного в ЭОР. Процесс усвоения образовательного контента отдельных компонентов ЭОР опирается на современные методы научного познания: эксперимент, сравнение, наблюдение, абстрагирование, обобщение, конкретизация, аналогия, индукция и дедукция, анализ и синтез, метод моделирования, в том числе и математического, а также метод системного анализа. 

3.2. Доступность обучения, осуществляемого с использованием ЭОР, означает соответствие теоретической сложности и глубины изучения образовательного контента возрастным и индивидуальным особенностям обучаемых.
3.3. Проблемность обучения обусловлена самой сущностью и характером учебно-познавательной деятельности. Уровень выполнимости этого дидактического требования может быть значительно выше при условии использования образовательного контента отдельных компонентов ЭОР по сравнению с использованием традиционных средств обучения.
3.4. Наглядность обучения означает необходимость восприятия образовательного контента отдельных компонентов ЭОР, их макетов или моделей и их личное наблюдение обучаемыми. Требование обеспечения наглядности в случае электронных ресурсов ЭОР. Распространение систем виртуальной реальности позволяет реализовать это требование на принципиально новом, более высоком уровне.
3.5. Сознательность обучения и активизации деятельности обучаемых предполагает обеспечение самостоятельных действий обучаемых при работе с образовательным контентом отдельных компонентов ЭОР. При этом мотивы деятельности обучаемых должны быть адекватны содержанию образовательного контента. Для повышения активности обучения ЭОР должен генерировать разнообразные учебные ситуации, формулировать разнообразные вопросы, предоставлять обучаемому возможность выбора той или иной траектории обучения, возможность управления ходом событий.
3.6. Системность и последовательность обучения при использовании
образовательного контента ЭОР обеспечивает последовательное усвоения
определенной системы знаний в изучаемой предметной области. Для этого
необходимо, чтобы знания, умения и навыки формировались в определенной системе, в строго логическом порядке.
3.7. Единство образовательных, развивающих и воспитательных функций обучения при использовании образовательного контента ЭОР.
Кроме традиционных дидактических требований к ЭОР предъявляются
специфические дидактические требования, обусловленные использованием
преимуществ современных образовательных ИКТ в создании и функционировании компонентов ЭОР.
3.8. Адаптивность подразумевает приспособляемость образовательного
контента ЭОР к индивидуальным возможностям обучаемых.
3.9. Интерактивность обучения предполагает взаимодействие обучаемых
с образовательным контентом ЭОР в процессе обучения. При этом средства Ларин С.Н., Герасимова Е.В. Критерии оценки эффективности
89
образовательного контента ЭОР должны обеспечивать диалог и обратную
связь в форме контроля действий обучаемых, вырабатывать рекомендации по
дальнейшей организации образовательного процесса и обеспечивать постоянный доступ к справочной информации.
3.10. Развития интеллектуального потенциала обучаемого при работе с образовательным контентом ЭОР предполагает формирование стиля мышления,
умения принимать оптимальное решение или вариативные решения в сложной
ситуации, умений обработки информации на основе использования систем
обработки данных, информационно-поисковых систем, баз данных и пр.
3.11. Структурно-функциональная взаимосвязь представления образовательного контента предполагает его четкую структуризацию и систематизацию в разрезе отдельных компонентов ЭОР.
3.12. Непрерывность дидактического цикла обучения означает, что образовательный контент ЭОР должен предоставлять возможность выполнения всех звеньев дидактического цикла в пределах одного сеанса работы с
образовательными ИКТ.
4. Методические требования.
С дидактическими требованиями к образовательному контенту ЭОР
тесно связаны методические требования, которые предполагают учет своеобразия и особенности конкретной предметной области, специфики соответствующей науки, ее понятийного аппарата, особенности методов исследования ее закономерностей, а так же возможностей реализации современных методов обработки информации и методологии образовательной деятельности. Образовательный контент ЭОР должен удовлетворять следующим методическим требованиям [7, с. 190].
4.1. Предъявление образовательного контента в ЭОР должно строиться
с опорой на взаимосвязь и взаимодействие понятийных, образных и действенных компонентов мышления.
4.2. Образовательный контент ЭОР должен отражать систему научных
понятий конкретной учебной дисциплины в виде иерархической структуры,
на каждом уровене которой обеспечивается учет как одноуровневых, так и
межуровневых логических взаимосвязей этих понятий.
4.3. Образовательный контент ЭОР должен предоставлять обучаемому
возможность контролируемых тренировочных действий с целью поэтапного
повышения уровня абстракции и усвоения знаний для осуществления обучаемыми алгоритмической и эвристической деятельности.
5. Психологические требования.
Наряду с учетом дидактических требований к разработке и использованию образовательного контента ЭОР выделяют ряд психологических требований, влияющих на успешность и качество их создания.
5.1. Представление образовательного контента в ЭОР должно соответствовать не только вербально-логическому, но и сенсорно-перцептивному и
представленческому уровням когнитивного процесса.90
АКТУАЛЬНЫЕ ВОПРОСЫ ПСИХОЛОГИИ И ПЕДАГОГИКИ
5.2. Представление образовательного контента в ЭОР должно иметь
свой тезаурус и быть ориентировано на лингвистическую композицию конкретного возрастного контингента обучаемых и специфику его подготовки.
5.3. Образовательный контент ЭОР должен быть направлен на развитие
образного и логического мышления.
Однако, как и в случае с технико-технологическими требованиями психологические требования так же относятся к сфере деятельности квалифицированных в этой области специалистов. Поэтому их дальнейшая конкретизация останется за рамками данного исследования.
6. Дизайн-эргономические и эстетические требования.
Эргономические требования к образовательному контенту ЭОР строятся с учетом возрастных особенностей обучаемых, обеспечивают повышение
уровня мотивации к профильному Интернет-обучению, устанавливают требования к изображению информации и режимам работы ЭОР. Основным
эргономическим требованием является требование организации в ЭОР и его
компонентах дружественного интерфейса, обеспечения возможности использования обучаемыми необходимых подсказок и методических указаний,
свободной последовательности и темпа работы, что позволит создать благоприятную атмосферу на занятиях [7, с. 203].
Эстетические требования тесно связаны с эргономическими требованиями и устанавливают соответствие эстетического оформления функциональному назначению ЭОР, упорядоченность и выразительность графических и изобразительных элементов учебной среды, соответствие цветового
колорита назначению ЭОР.
Соблюдение приведенной выше системы требований и критериальных
показателей для оценки эффективности образовательного процесса на основе
применения современных ИКТ, их дидактического контента и разрабатываемых на их основе ЭОР позволит не только выявить, но и правильно учесть все
особенности подготовки и представления содержания образовательного процесса на этапах проектирования, разработки, апробации и практического внедрения современных образовательных ИКТ и ЭОР в образовательной сфере.
Большая часть современных ИКТ относится к классу прорывных технологий, обеспечивающих быстрое накопление интеллектуального и экономического потенциала и гарантирующих устойчивое развитие общества.
Внедрение современных ИКТ в образовательный процесс приводит к коренному изменению функций педагога, который становится исследователем, организатором, консультантом, в том числе в информационном пространстве.
Применение современных ИКТ ведет к развитию и преобразованию деятельности человека практически любого возраста за счет возникновения новых
навыков, операций, процедур и способов выполнения действий, новых целевых и мотивационно-смысловых структур, новых форм опосредования и
совсем новых видов деятельности в информационной среде. Использование Ларин С.Н., Герасимова Е.В. Критерии оценки эффективности
91
ИКТ способствует: индивидуализации учебно-воспитательного процесса с
учетом уровня подготовленности, способностей и индивидуально-типологических особенностей усвоения материала, интересов и потребностей обучаемых; изменению характера познавательной деятельности обучаемых в
сторону ее большей самостоятельности и поискового характера; стимулированию стремления обучаемых к постоянному самосовершенствованию и
готовности к самообразованию; расширению возможностей установления и
проработки междисциплинарных связей; повышению гибкости, мобильности учебного процесса, его постоянному и динамичному обновлению.

Это достигается путём кропотливой тврорческой работы, многочисленных проверок. Соответствие учебника дидактическим требованиям ложится на автора учебного пособия. Большенство бумажных учебников прошли проверку на соответствие дидактическим требованиям и одобрены для использования в качестве учебников. Что касается электронных учебников, то смело можно сказать, что если это просто pdf бумажного учебника, то он, при соответствии дидактическим требованиям не соответствует функциональным, поскольку не использует потенциала электронных устройств. Он не интерактивен, в нем нет аудио и видео материалов, анимации, схем, поэтому он менее удобен чем бумажный аналог.