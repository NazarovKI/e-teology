\large
\section{ЭОР в преподавании теологических дисциплины}
Педагогический потенциал представляет собой свойство, присущие социально значимым предметам и явлениям, посредством которых становится возможным формирование и становление личности в процессе её образования. Ряд ученых постепенно пришли к консенсусу о том, что для достижения максимального эффекта влияния электронных ресурсов в образовательной среде требуется согласование связей между всеми компонентами технологии (аппаратное и программное обеспечение), образовательным контекстом и пользователями (преподавателями и студентами)\cite{ivinskaya22}.

Курс ОРКСЭ – составная часть единого образовательного пространства духовно-нравственного развития и воспитания обучающегося, включающего урочную, внеурочную, внешкольную и общественно полезную деятельность.

Цель учебного курса ОРКСЭ – формирование у младшего подростка мотиваций к осознанному нравственному поведению, основанному на знании и уважении культурных и религиозных традиций многонационального народа России, а также к диалогу с представителями других культур и мировоззрений.

%Задачи учебного курса ОРКСЭ:
%\begin{enumerate}
%\item знакомство обучающихся с основами православной, мусульманской, буддийской, иудейской культур, основами мировых религиозных культур и светской этики;
%\item развитие представлений младшего подростка о значении нравственных норм и ценностей для достойной жизни личности, семьи, общества;
%\item обобщение знаний, понятий и представлений о духовной культуре и морали полученных обучающимися в начальной школе, и формирование у них  ценностно-смысловых мировоззренческих основ, обеспечивающих целостное восприятие отечественной истории и культуры при изучении гуманитарных предметов на ступени основной школы;
%\item развитие способностей младших школьников к общению в полиэтнической и многоконфессиональной среде на основе взаимного уважения и диалога во имя общественного мира и согласия.
%\end{enumerate}
\subsection{Педагогический потенциал ЭОР}
\label{section2}
Для выявления педагогического потенциала ЭОР необходимо рассмотреть дидактические трудности, возникающие в преподавании ОРКСЭ и сопоставить их с дидактическими возможностями электронных образовательных ресурсов, понять насколько последние помогают в решении первых.

%Для выявления  \hyperref[goal]{педагогического потенциала} ЭОР в преподавании теологических дисциплин необходимо определить место и функцию теологии в общей структуре школьных предметов. 
%В системе образования теология, выполняет важную мировоззренческую и воспитательную функции, имеет тесную связь с литературой, историей, философией и другими предметами, формирующими мировоззрение человека, помогает установить связи между естественными и гуманитарными науками.

В процессе преподавания ОРКСЭ у педагогов возникают следующие дидактические трудности:

В преподавании ОРКСЭ используется культурологический подход, который нацелен на формирование у школьников представления о религии как о важнейшей составляющей мировой культуры.

%В парадигме цифровизации образования меняются условия коммуникации, учитель перестаёт быть «транслятором знаний», ведущими направлениями деятельности преподавателя становится сопровождение и поддержка, создание условий для общения.

Учебная деятельность становится метапредметной, поскольку жизненный опыт показывает, что решение любой практической задачи всегда требует привлечения знаний из нескольких предметных областей поэтому необходимо внимательно отнестись к развитию межпредметных связей в образовании\cite{metapredmet}.

%Существует три уровня межпредметных связей: 1) внутри предметов одного цикла (например, ботаника -- биология); 2) внутри предметов естественнонаучного цикла (биология и физика, педагогика и психология); 3) показать обучаемым связь частных понятий с философскими категориями.

%Для реализация межпредметного взаимодействия требуется определить функции предмета в общей системе образования.

%Данные исследования помогут обеспечить должную интерактивность учебного процесса и помочь ученикам в организации самообучения и выбора направлений обучения\cite{osobennosti_eor}.

%Таким образом, для успешной интеграции в цифровую образовательную среду образовательный контент должен соответствовать принципам дидактики (быть доступным и наглядным; последовательно и систематично излагаться).
%Модули должны иметь ссылки на естественно-научные и гуманитарные предметы, создавая и поддерживая таким образом межпредметные связи и соответствовать высоким требованиям редакторской правки и верстки текста.



\begin{enumerate}
\item Для реализации культурологического подхода в преподавании ОРКСЭ пособие должно быть как можно более наглядным. Бумажные учебники не могут в полной мере обеспечить соблюдение принципа наглядности, поэтому их дополняют QR кодами, по которым можно получить задание. А в тетради остаются только письменные проверочные задания и тесты.

\item Коммуникационные технологии (презентации, видеоролики) т.к. иллюстративного материала и информации в учебнике недостаточно, к тому же презентации развивают зрительную память и учебный материал лучше усваивается.

\item Безотметочная система предполагает групповую активность детей на уроке поэтому необходимы дискуссионные формы работы. Например Просмотр фильма и организация дискуссии.

\item Трудность: Малое количество часов в классе ---  Решение: подготовка переносится на дом. 

В условиях перехода на смешанное обучение, традиционные аудиторные виды занятий переносятся во внеаудиторную, самостоятельную часть работы (дистант). Дети самостоятельно готовят сообщения к урокам, современные школьники, используя цифровые ресурсы, учатся искать информацию по поставленным вопросам, совершенствовать свои умения в переработке и предоставлении информации (подготовка рефератов, презентаций, сообщений докладов), между пользователями осуществляется сетевой обмен\cite{shevardina}. 

\item Трудность: Отсутствие квалифицированных Преподавателей. 
Решение: видеоуроки, вопросы  родителям.

\item В книге для учителя мало материала, непосредственно направляющего к разработкам уроков. Хотелось  бы иметь под рукой такую книгу, которая помогала бы учителю при составлении конспектов. Приходится использовать различные справочники, энциклопедии, что отнимает очень много времени. Вместе с тем хорошую помощь и поддержку при подготовке к урокам оказывают материалы, разработки  уроков размещённые  на интернет- сайтах. 

\end{enumerate}

Примеры заданий в рабочей тетради:
\begin{itemize}
\item филворд
\item кроссворды
\item найти соответствие между термином и его определением,
\item предлагаю ребятам самим выбрать задание и рассказать о нём все, что знают
\item тесты
\end{itemize}

%В качестве практического предложения можно внести небольшое улучшение в работу образовательного портала миссионерского института. А именно: выкладываем лекции только аудио, к ним будет проще получить доступ с телефона. Это делается мгновенно при помощи ffmpeg
\subsection{Образовательные задачи по теологии}


Какие существуют запросы от людей?
Поскольку создание ЭОР требует больших затрат времени труда авторов и программистов, необходимо вначале определить целесообразность создания ЭОР. В беседе с преподавателем воскресной школы выяснили, что для младших школьников целесообразно создать такие ресурсы по следующим темам:
Иконы пресвятой Богородицы явленные в России (в ХХ веке); – есть идея создать интегрированный урок который связывает области знаний Истории и ОПК.
Двунадесятые праздники;
Недели великого поста;
Страстная седмица;
Божественная литургия – желательный вид материала – короткие видео с пояснением священника, по ходу каждого этапа Божественной литургии.
Богослужебная утварь – для раскрытия темы достаточно фотографий с лаконичными текстовыми пояснениями.
ЭОР по дисциплине «Теория и история христианского искусства».